%%%%  CAVEAT:  xelatex chokes on pkg soul (loaded by tufte); use lualatex

%% \documentclass[12pt,toc]{tufte-handout}
\documentclass[reqno,12pt]{tufte-book}
%% \usepackage{trace}
%% \documentclass[reqno,12pt]{article}

\usepackage{draftwatermark}

% BLACK & WHITE
%% \input{opt-black-white}

%% \usepackage{etex}

%% %%%%%%%%%%%%%%%%%%%%%%%%%%%%%%%%%%%%%%%%%%%%%%%%%%%%%%%%%%%%%%%%
%% %% packages included by original hott main.tex

%% %%% For table {tab:theorems}
%% \usepackage{pifont}

%% %%% Multi-Columns for long lists of names
%% \usepackage{multicol}

%% \usepackage{graphicx}
%% \usepackage{comment}

%% \usepackage{fancyhdr} % To set headers and footers

%% %% \usepackage{nextpage} % So we can jump to odd-numbered pages

%% \usepackage{amssymb,amsmath,stmaryrd,mathrsfs,wasysym}
%% \usepackage{enumitem,mathtools,xspace}
%% %% \numberwithin{equation}{subsection}

%% \usepackage{xstring} % For generating singluars and plurals in \backref

%% %% \usepackage{xcolor,mdframed}
%% \usepackage{xcolor} % For colored cells in tables we need \cellcolor
%% \usepackage{wallpaper} % For the background image on the cover page

%% \usepackage{booktabs} % For nice tables
%% \usepackage{array} % For nice tables

%% %% \definecolor{linkcolor}{rgb}{\OPTlinkcolor}
%% \usepackage{aliascnt}
%% \usepackage[capitalize]{cleveref}
%% \usepackage[all,2cell,cmtip]{xy}
%% \UseAllTwocells
%% %\usepackage{natbib}
%% \usepackage{braket} % used for \setof{ ... } macro

%% \usepackage{tikz}
%% \usetikzlibrary{decorations.pathmorphing,arrows}

%% \usepackage{etoolbox}           % hacking commands for TOC

%% %% \usepackage{mathpartir}         % for formal.tex appendix, section 3

%% \usepackage[numbered]{bookmark} % add chapter/section numbers to the toc in the pdf metadata
%% %%%%%%%%%%%%%%%%%%%%%%%%%%%%%%%%%%%%%%%%%%%%%%%%%%%%%%%%%%%%%%%%

%% these two go together!
\usepackage{framed}
\usepackage[standard,framed]{ntheorem}
%% \newtheorem{theorem}{Theorem}
%\newtheorem{cor}{Corollary}
%\newtheorem{lem}{Lemma}
%% \newtheorem*{defn}{Definition}
%% \theoremstyle{remark}
%% \newtheorem{remark}{Remark}
%% \newtheorem*{commentary}{Commentary}

%% \theoremclass{Remark}
%% \theoremstyle{break}
%% \newtheorem{note}{Note}[section]

\theoremstyle{plain}
\theorembodyfont{\upshape}
\theoremsymbol{\ensuremath{\ast}}
\theoremseparator{}
%% \newtheorem{ednote}{Ed. note}[section]
\newframedtheorem{ednote}{Ed. note}[section]

\newtheorem*{todo}{TODO}
%% \newtheorem{eg}{Example}

%% \input{macros}

\usepackage{appendix}

%% \usepackage{csquotes}

\usepackage{setspace}

%% broken (doesn't work with tufte-handout):
\usepackage{zed-csp}
%% broken:
%% \usepackage{ltcadiz-fam}

\usepackage{fontspec}
%% \usepackage{xltxtra,xunicode}
\defaultfontfeatures{Scale=MatchLowercase}

%% \defaultfontfeatures{Scale=MatchLowercase}
%% \setmainfont[Mapping=tex-text]{Times New Roman}
%% \setsansfont[Mapping=tex-text]{Arial}
%% \setmonofont{Courier}

\setmainfont[Ligatures=TeX]{TeX Gyre Bonum}
\setromanfont[Ligatures=TeX]{TeX Gyre Bonum}
\setsansfont[Ligatures=TeX]{TeX Gyre Adventor}
\setmonofont[Ligatures=TeX]{TeX Gyre Cursor}


%% \setmainfont[Mapping=tex-text]{Minion Pro}
%% \setromanfont[Mapping=tex-text]{Minion Pro}
%% \setsansfont[Mapping=tex-text]{TeX Gyre Heros}

%% Bugfix: see https://code.google.com/p/tufte-latex/issues/detail?id=64
% Set up the spacing using fontspec features
\renewcommand\allcapsspacing[1]{{\addfontfeature{LetterSpace=15}#1}}
\renewcommand\smallcapsspacing[1]{{\addfontfeature{LetterSpace=0.0}#1}}

\usepackage{epigraph}
\setlength{\epigraphwidth}{.8\textwidth}

%% general symbols - degree, etc.
%% \usepackage{gensymb}

\usepackage [english]{babel}
\usepackage [autostyle, english = american]{csquotes}
%% \usepackage{quoting}

%% nice double-stroke fonts
\usepackage{dsfont}

% Small sections of multiple columns
\usepackage{multicol}

% Provides paragraphs of dummy text
\usepackage{lipsum}

% The units package provides nice, non-stacked fractions and better spacing
% for units.
\usepackage{units}

%\usepackage{geometry}                % See geometry.pdf to learn the layout options. There are lots.
%\geometry{letterpaper}                   % ... or a4paper or a5paper or ...

\usepackage{xfrac}

\usepackage{hyperref}
\hypersetup{
  bookmarks=true,         % show bookmarks bar?
  bookmarksdepth=3,
  unicode=true,          % non-Latin characters in Acrobat’s bookmarks
  pdftoolbar=true,        % show Acrobat’s toolbar?
  pdfmenubar=true,        % show Acrobat’s menu?
  pdffitwindow=false,     % window fit to page when opened
  pdfstartview={FitH},    % fits the width of the page to the window
  pdftitle={Intuition and Exponentiation},    % title
  pdfauthor={G. A. Reynolds},     % author
  pdfsubject={Mathematics},   % subject of the document
  pdfcreator={G. A. Reynolds},   % creator of the document
  pdfproducer={Producer}, % producer of the document
  pdfkeywords={Exponentiation} {Mathematics}
  pdfnewwindow=true,      % links in new window
  colorlinks=true,       % false: boxed links; true: colored links
  linkcolor=blue,          % color of internal links
  citecolor=blue,        % color of links to bibliography
  filecolor=magenta,      % color of file links
  urlcolor=cyan           % color of external links
}

%% \usepackage[
%% bibstyle=numeric,
%% citestyle=authoryear,
%% hyperref,
%% bibencoding=utf8,
%% backref=true,
%% backend=biber]{biblatex}

%% http://tex.stackexchange.com/questions/66778/citation-alias-with-multibib-and-natbib
%% \makeatletter
%% \def\@mb@citenamelist{cite,citep,citet,citealp,citealt,citepalias,citetalias}
%% \makeatother

%% http://stackoverflow.com/questions/2496599/how-do-i-cite-the-title-of-an-article-in-latex
\defcitealias{z-iso-13568}{ISO 13568:2002 Information technology -- Z formal specification notation --
  Syntax, type system and semantics}

\usepackage{tikz}
\usepackage[markings,customcolors]{hf-tikz}
\usetikzlibrary{%
  arrows%
  ,calc%
  ,decorations.text%
  ,decorations.pathreplacing%
  ,fadings%
  ,positioning
  ,shapes.geometric%
}

\usepackage{tikz-3dplot}

\usepackage{pgfplots}
\pgfplotsset{height=7cm,compat=1.9}

\usepackage{tkz-euclide}
\usetkzobj{all}

%% prettier integral syms, but broken on miktex
%% \usepackage{esint}


%% \usepackage{MnSymbol}
%% \usepackage[misc]{ifsym}

%% \usepackage{morefloats}

%%%%%%%%%%%%%%%%%%%%%%%%%%%%%%%%%%%%%%%%%%%%%%%%%%%%%%%%%%%%%%%%
\title{Dependent Types}
%% \\
%% \Large Derived from the HoTT Book}
\author{}
%\date{}                                           % Activate to display a given date or no date

%%%%%%%%%%%%%%%%
%% tufte-latex customizations

\makeatletter
\let\runauthor\@author
\let\runtitle\@title
\makeatother

%% running headers
\newcommand{\changefont}{%
  \fontsize{7}{9.5}\selectfont
}
\fancypagestyle{plain}{
  \fancyhead[LO,LE]{\leftmark }
  \fancyhead[RO,RE]{\rightmark}
  \fancyfoot[CO,CE]{\thepage}
  \fancyfoot[LE]{\textsc{\runtitle}}
  \fancyfoot[RO]{\textsc{\runtitle}}
  \renewcommand{\headrulewidth}{0pt}
  \renewcommand{\footrulewidth}{0pt}
}
\pagestyle{plain}

\def\chpcolor{blue!45}
\def\chpcolortxt{blue!60}
\def\sectionfont{\LARGE}

\setcounter{secnumdepth}{5}
\setcounter{tocdepth}{5}        % sections and subsections for the toc

\makeatletter
%% Section:
\def\@sectionstrut{\vrule\@width\z@\@height12.5\p@}
\def\@makesectionhead#1{%
  {%\par\vspace{20pt}%
    \parindent -10pt\raggedleft\sectionfont
    %% \colorbox{\chpcolor}{%
    %%   \parbox[t]{90pt}{\color{white}\@sectionstrut\@depth4.5\p@\hfill
    %%     \ifnum\c@secnumdepth>\z@\thesection\fi}%
    %% }%
    \vspace{10pt}%
    \begin{minipage}[t]{\textwidth}%{\dimexpr\textwidth-90pt-2\fboxsep\relax}
      \@sectionstrut\hspace{-15pt}\textit{\textbf\Huge #1}
    \end{minipage}\par
    \vspace{5pt}%
  }
}
%% \def\@makesectionhead#1{%
%%   {\par\vspace{20pt}%
%%    \parindent 0pt\raggedleft\sectionfont
%%    \colorbox{\chpcolor}{%
%%      \parbox[t]{90pt}{\color{white}\@sectionstrut\@depth4.5\p@\hfill
%%        \ifnum\c@secnumdepth>\z@\thesection\fi}%
%%    }%
%%    \begin{minipage}[t]{\dimexpr\textwidth-90pt-2\fboxsep\relax}
%%    \color{\chpcolortxt}\@sectionstrut\hspace{5pt}\textbf{#1}
%%    \end{minipage}\par
%%    \vspace{10pt}%
%%   }
%% }
\def\section{\@afterindentfalse\secdef\@section\@ssection}
\def\@section[#1]#2{%
  \ifnum\c@secnumdepth>\m@ne
  \refstepcounter{section}%
  \addcontentsline{toc}{section}{\protect\numberline{\thesection}#1}%
  \else
  \phantomsection
  \addcontentsline{toc}{section}{#1}%
  \fi
  \sectionmark{#1}%
  \if@twocolumn
  \@topnewpage[\@makesectionhead{#2}]%
  \else
  \@makesectionhead{#2}\@afterheading
  \fi
}
\def\@ssection#1{%
  \if@twocolumn
  \@topnewpage[\@makesectionhead{#1}]%
  \else
  \@makesectionhead{#1}\@afterheading
  \fi
}
\makeatother

%%%%%%%%%%%%%%%%
%% macros

\newcommand\tok[1]{%
\(\llcorner #1\lrcorner\)
}
\newcommand\typ[1]{%
\(\ulcorner #1\urcorner\)
}

\newenvironment{important}[1][]{%
  \begin{mdframed}[%
      backgroundcolor={red!15}, hidealllines=true,
      skipabove=0.7\baselineskip, skipbelow=0.7\baselineskip,
      splitbottomskip=2pt, splittopskip=4pt, #1]%
    \makebox[0pt]{% ignore the withd of !
      \smash{% ignor the height of !
        \fontsize{32pt}{32pt}\selectfont% make the ! bigger
        \hspace*{-19pt}% move ! to the left
        \raisebox{-2pt}{% move ! up a little
          {\color{red!70!black}\sffamily\bfseries !}% type the bold red !
        }%
      }%
    }%
}{\end{mdframed}}

%% reversed integral sign
\makeatletter
\providecommand*{\curv}{%
  \mathrel{%
    \mathpalette\@curv\int
  }%
}
\newcommand*{\@curv}[2]{%
  \reflectbox{$\m@th#1#2$}%
}
\makeatother

%% \def\LaTeX{%
%%   L\kern-.36em
%%   {\setbox0=\hbox{T}%
%%     \vbox to \ht0{\hbox{\the\scriptfont0 A}\vss}}%
%%   \kern-.15em
%%   \TeX
%% }

%%%%%%%%%%%%%%%%

\newcommand\cspace{coordinate space}
\newcommand\Cspace{Coordinate space}
\newcommand\CSpace{Coordinate Space}

\newcommand\dspace{design space}
\newcommand\Dspace{Design space}
\newcommand\DSpace{Design Space}

\newcommand\Omg{\(\Omega\)}
\newcommand\sccs{standard cartesian coordinate space}
\newcommand\origin{\((0,0)\)}
\newcommand\ab{\((a,b)\)}

\newcommand\atypeA{\ensuremath{(a : A)}}

%% \newcommand\N{\(\mathds{N}\)}
%% \newcommand\R{\(\mathds{R}\)}
%% \newcommand\RR{\(\mathds{R}\times\mathds{R}\)}
%% \newcommand\Rtwo{\(\mathds{R}^2\)}
%% \newcommand\Z{\(\mathds{Z}\)}


\def\HoTT{%
  H\kern-.7pt
  {\tiny\raisebox{1pt}{o}}%
  %% {\setbox0=\hbox{T}%
  %%  \vbox to \ht0{\vss\hbox{\the\scriptfont0 o}\vss}}%
  \kern-1.5pt
  TT}

\def\HoTTB{%
  the H\kern-.7pt
  {\tiny\raisebox{1pt}{o}}%
  %% {\setbox0=\hbox{T}%
  %%  \vbox to \ht0{\vss\hbox{\the\scriptfont0 o}\vss}}%
  \kern-1.5pt
  TT Book
}

\newcommand\ML{Martin-L\"{o}f}

\newcommand\ITT{Intuitionisti Type Theory}

\newcommand\TTh{Type Theory}
\newcommand\tth{type theory}

\includeonly{%
types
,dtypes
,terms
}

%%%%%%%%%%%%%%%%%%%%%%%%%%%%%%%%%%%%%%%%%%%%%%%%%%%%%%%%%%%%%%%%
\begin{document}
%% \ifx\traceon\undefined \tracingall \else \traceon \fi

\maketitle

\begin{ednote}
Dependent types ...
\end{ednote}

\tableofcontents
%% \setcounter{tocdepth}{2}        % chapters, sections, and subsections for the
%%                                 % metadata of the pdf
%% \cleartooddpage[\thispagestyle{empty}]

%% \mainmatter % Turn on roman page numbers and numbered chapters

%%%%%%%%%%%%%%%%%%%%%%%%%%%%%%%%
\chapter{Introduction}

Intro ...

\chapter{Types}

\section{Type as Concept}
A type is a concept, and concepts are what we use to categorize the
world.

Chibi, cat, feline, mammal.

\section{Type as Syntactic Category}

\section{Type as Rule}

\section{Example}

Integers.  A set under the classic description, a type under type theory.

How can we ``construct'' integers?  Classic: how can we come up with
expressions that \textit{denote values}?  Type theory: how can we
\textit{construct terms}?

Numerals.  It's easy to define construction of single-digit numerals:
just list the digits:

\begin{verbatim}

DecimalNumeral:  0 | 1 | .. | 9 ;

\end{verbatim}

Compare this to the metasyntax of an ordinary grammar (e.g. ANTLR4):

\begin{verbatim}

DecimalNumeral:  '0' | '1' | .. | '9' ;

\end{verbatim}

We have to quote the ``constructors'', because we're dealing with a
host language.

The constants so declared count as constructors.  Using them
constructs a term.  Classically, we don't have the notion of
constructing a term; instead, we have the idea of writing a
grammatical sentence that denotes a value.  The grammar is
antecedently defined; our DecimalNumeral production is part of the
language definition, not a program.  In type theory, the type
declaration is not part of the grammar of the language, but is a
program.  It implicitly defines a grammar; we view ``DecimalNumeral''
as a syntactic category whose ``members'' are 0, 1,..,9.

Ok, so how can we construct longer numerals, like '123'?

Classically, in the syntax definition.  Here's how ANTLR4 does it for Java:

\begin{verbatim}
DecimalNumeral
    :   '0'
    |   NonZeroDigit (Digits? | Underscores Digits)
    ;

fragment
Digits
    :   Digit (DigitOrUnderscore* Digit)?
    ;

fragment
Digit
    :   '0'
    |   NonZeroDigit
    ;

fragment
NonZeroDigit
    :   [1-9]
    ;

fragment
DigitOrUnderscore
    :   Digit
    |   '_'
    ;

fragment
Underscores
    :   '_'+
    ;

\end{verbatim}

(Why ``NonZeroDigit'' etc.?  Because e.g. HexNumerals start with 0;
this allows a standalone 0 to count as a decimal numeral.)

This bit of grammar defines the form of decimal numerals, but it says
nothing about how to ``construct'' them.  By that I mean there is no
procedure associated with the grammatical rule.

By contrast, a type-theoretic definition might look something like this:

\begin{verbatim}
Declare DecimalNumeral
    := 0 | 1 | 2 | 3 | 4 | 5 | 6 | 7 | 8 | 9
    | 0 DecimalNumeral
    | 1 DecimalNumeral
    | 2 DecimalNumeral
    | 3 DecimalNumeral
    | 4 DecimalNumeral
    | 5 DecimalNumeral
    | 6 DecimalNumeral
    | 7 DecimalNumeral
    | 8 DecimalNumeral
    | 9 DecimalNumeral
    ;
\end{verbatim}

We can simplify this a bit by using a helper:

\begin{verbatim}
Declare DecimalDigit := 0 | 1 | 2 | 3 | 4 | 5 | 6 | 7 | 8 | 9 ;
Declare DecimalNumber := DecimalDigit | DecimalDigit DecimalNumber ;
\end{verbatim}

Note that these are recursive definitions.  In most grammar
meta-syntaxes, we might write something like:

\begin{verbatim}
Declare DecimalNumber := DecimalDigit+ ;
\end{verbatim}

But this relies on the grammatical meta-symbol '+', which is
unavailable.

Now something like this could be written in a grammatical meta-syntax
as part of a language definition.  But we make it part of the
language, something the user can write.  It has the effect of making
the decimal digits constructors.  An expression like ``23'' is
interpreted as applying the constructor ``2'' to the result of
applying ``3''.

Neither of these defines meaning.  For that we need additional rules,
etc.  In most programming languages, the interpretation of numerals is
implicitly defined by the compiler implementation.  In proof
assistants and some functional programming languages, this can be
specified using the language, as part of a ``prelude'' library.

In other words, type-theoretic languages can be absolutely minimal,
with no predefined semantics and hardly any syntax.  Type declarations
and definitions etc. then extend the language.

\begin{ednote}
  What makes this construction?
\end{ednote}

\begin{ednote}
  Anyway the point is the connection between type-theoretic programs
  and grammatical specification.
\end{ednote}



\chapter{Dependent Types}

\section{Symbols}

We're going to use numbers to explain dependent types.  We'll be
working in base 10, so we need 10 decimal digits.  Of course we can
use the standard digits 0-9, but we don't have to; we can use any
symbol at all.  For example:

{ ○ , △ , □ , ♡ , ♤ , ♢ , ♧ , ☆ , ♀ , ♂ }


Nor need we restrict our set to 10 symbols; we could stipulate that
our symbol set is infinite, but to actually use it we have to be able
to write them all down, which means we have to have a finite set, but
it can be as large as we want.  Of course the usual way to proceed is
to define a minimal set of ten symbols together with a
numeral-formation rule that allows us to construct an infinite number
of symbols from a finite set by simple concatenation of strings of the
basic ten: 1234, for example.

In fact we only need a single symbol; in what follows we'll show how
one symbol can be used to define (construct) any number system.

We'll call our basic symbol set the ``alphabet'' of our language.
We're already calling the elements of the set symbols, which means we
already have an implicit type: the concept ``symbol'' categorizes the
elements of our alphabet set.  We're going to need the term ``symbol''
later, so lets rename this category to ``symbolic'', so we can say
that our alphabet is a set of symbolics:

Symbolic :: { 0, 1, 2, 3, 4, 5, 6, 7, 8, 9 }

alternatively:

Symbolic :: { ○ , △ , □ , ♡ , ♤ , ♢ , ♧ , ☆ , ♀ , ♂ }




Example: set of Ints.  We can recognize 1 and 2 as integers and \{\}
as a set; more precisely, we can recognize the symbols because we're
familiar with the conventions governing their usage and
interpretation.  So we can recognize that the form \( \{1, 2\}\)
expresses (denotes) a set of integers.

Note that the concept ``set of integers'' is a complex concept that
depends on two more basic concepts, namely ``set'' and ``integer''.
So the more complex concept may be said to \textit{depend} on its more
basic component concepts.

The critical point here is that the combination of ``set'' and
``integer'' goes both ways.  It may seem most ``natural'' to recognize
\( \{1, 2\}\) as an example of the category (type) ``set of
integers'', but that's a matter of convention; it is equally
legitimate to recognize it as an element of the category ``integer of
sets''.  Not ``integer of sets of integers'', mind you, but just
``integer of sets''.  To see this, all that is needed is a little
reflection on the concepts involved.  First of all, the
\textit{symbols} 1 and 2 are not themselves integers; nor do the
reveal anything about the nature of integers.  The same considerations
apply to symbolic forms like \(\{1, 2\}\).  These forms are just
symbolic forms, and any form can play the role of symbol.  In
particular, sets -- or rather, set forms like \( \{1, 2\}\) -- can be
treated as symbols of anything we like.

Furthermore, integers and sets themselves -- rather than the symbolic
forms we use to refer to them -- can serve as symbols.  There is
nothing strange about calling a set an integer, of course; according
to standard set theory, that is precisely what integers are, special
kinds of sets.

The larger point: we never have direct access to things; all we have
is various descriptions.  The same thing under different descriptions
can look like different things.  That is, when we categorize things we
do not really categorize the thing itself; rather, we categorize it
\textit{under a description}.  Or, to categorize is to describe.  So a
set may ``look like'' a mere set under one description; under another
description, it comes out as an integer.

Back to the notion of ``integer of sets''.  This counts as a dependent
type.  It's a member of an \textit{indexed family of types}; other
members of this family include ``rational of sets'', ``real of sets'',
etc.

In other words, if we can for type X of Y, we can also form the type Y
of X.

What do the elements of type ``integer of sets'' look like?  Just like
sets, in fact, with one difference: the elements themselves must be
typed, which means that we have to use a \textit{constructor} for form
expressions of the type.  Let's declare that \texttt{intset} is the
constructor; then e.g. \texttt{intset \{1, 2\}} is a term of type
``integer of sets''.  (Of course we could define other syntax,
e.g. \texttt{intset 1 2}; we retain the braces to stress the concept
of integer of sets.)

So ``integer of sets'' is a type; that is, it categorizes some things.
What are those things?  Integers that depend on sets.  The type
``integer of X'' is a type family; its individuals depend on X, so
``integer of sets'' depends on ``set'', just as ``set of integers''
depends on ``integer''.

Some Int of Sets terms: intset \{1, 2\}; intset \{foo, bar\}; inset \{
\{\}, \{\{\}\} \}.  Note that these terms have something in common:
they all contain two elements.

So under the right description, these things can be treated as
integers.  For example, Int-of-Set elements of size two form an
equivalence class; the integer '2' corresponds to that class.  Note
that this makes individual integers correspond to sets of intsets.

Replace ``set'' by ``sequence'' and you get a variation on the same
theme.  An Int-of-Sequences is a kind of int that depends on the type
``sequence'' rather than set, just as ``Sequence-of-Int'' is a kind of
sequence that depends in Int (rather than, say Rational or Real).

Anyway, the point of all this is that thinking in (dependent) types is
very flexible; what counts as a type depends on how you want to
conceptualize things.

In fact, the ``base'' type of a dependent type of this kind doesn't
really matter.  A type ``X of Y'' is a new type that is distinct from
by X and Y, although it is dependent on both.  So it isn't really a
``kind'' of either X or Y.  A List-of-Int is not really a kind of
List, nor a kind of Int.  To say that one type is dependent on another
is not to say that it is a type of that kind.  A depends on B != A is
a kind of B.  E.g. List-of-Int: its ``kind'' is ``List-of-A'' where A
is a type variable (schematic var), not ``List''.  It's an instance of
the schema.

What kind a thing it is also depends on ops?  Algebra.


\chapter{Terms}

Instead of ``type'' we talk only of terms.  The key is to think in terms
of ``orders''.  A term of order n serves to categorize some set of terms
of order n-1.  For example, Chibi is my housecat.  That makes ``Chibi''
a term of order 0; it categorizes nothing except the individual it
names.  ``Housecat'' is a term of order 1; it categorizes individuals
that are cats.  ``Feline'' is a term of order 2; it categorizes terms of
order 1, like ``housecat'', ``big cat'', etc.  ``Mammal'' is a term of order
3; it categorizes terms of order 2, like ``Feline'', ``Canine'', ``Ape'', etc.

So numerals like 0, 1, 327, etc. are terms or order 0; they categorize
only themselves (or maybe the ``values'' they denote).  Type terms like
``Natural'', ``Integer'', etc. are terms of order 1: they categorize the
numerals.  Dependent types like ``List of Int'' are also terms of order
1: they categorize the terms of order 0 involving constructors, like
``3::2::1::nil'' (in Haskell-like notation).

Dependent type families like ``List of X'' are terms of order 2: they
categorize the order 1 terms like ``List of Int'', ``List of
Rational'', etc.

So ``type'' in standard terminology corresponds to ``2nd order term''
in our terminology.  Keeping track of abstraction -- answering
questions like ``what is the type of a type?'' -- is a lot easier
using this terminology.  The type of any term of order n is a term of
order n+1.


\chapter{The Hierarchy of Terms}

\clearpage
\appendix
\begin{appendices}
  %%%%%%%%%%%%%%%%%%%%%%%%%%%%%%%%%%%%%%%%%%%%%%%%%%%%%%%%%%%%%%%%

  %%%% Bibliography
  %% \bibliographystyle{halpha}
  %% \phantomsection % black magic to get TOC to point to correct page
  %% \addcontentsline{toc}{part}{\bibname}
  %% \markboth{}{\textsc{Bibliography}}
  %% {\renewcommand{\markboth}[2]{} % Prevent bibliography from resetting the header to something silly
  %% \OPTbibliographyfont

  
  %% \bibliography{references}
  %% \bibliographystyle{plainnat}

\end{appendices}

\end{document}
