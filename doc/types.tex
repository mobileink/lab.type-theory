\chapter{Types}

\section{Type as Concept}
A type is a concept, and concepts are what we use to categorize the
world.

Chibi, cat, feline, mammal.

\section{Type as Syntactic Category}

\section{Type as Rule}

\section{Example}

Integers.  A set under the classic description, a type under type theory.

How can we ``construct'' integers?  Classic: how can we come up with
expressions that \textit{denote values}?  Type theory: how can we
\textit{construct terms}?

Numerals.  It's easy to define construction of single-digit numerals:
just list the digits:

\begin{verbatim}

DecimalNumeral:  0 | 1 | .. | 9 ;

\end{verbatim}

Compare this to the metasyntax of an ordinary grammar (e.g. ANTLR4):

\begin{verbatim}

DecimalNumeral:  '0' | '1' | .. | '9' ;

\end{verbatim}

We have to quote the ``constructors'', because we're dealing with a
host language.

The constants so declared count as constructors.  Using them
constructs a term.  Classically, we don't have the notion of
constructing a term; instead, we have the idea of writing a
grammatical sentence that denotes a value.  The grammar is
antecedently defined; our DecimalNumeral production is part of the
language definition, not a program.  In type theory, the type
declaration is not part of the grammar of the language, but is a
program.  It implicitly defines a grammar; we view ``DecimalNumeral''
as a syntactic category whose ``members'' are 0, 1,..,9.

Ok, so how can we construct longer numerals, like '123'?

Classically, in the syntax definition.  Here's how ANTLR4 does it for Java:

\begin{verbatim}
DecimalNumeral
    :   '0'
    |   NonZeroDigit (Digits? | Underscores Digits)
    ;

fragment
Digits
    :   Digit (DigitOrUnderscore* Digit)?
    ;

fragment
Digit
    :   '0'
    |   NonZeroDigit
    ;

fragment
NonZeroDigit
    :   [1-9]
    ;

fragment
DigitOrUnderscore
    :   Digit
    |   '_'
    ;

fragment
Underscores
    :   '_'+
    ;

\end{verbatim}

(Why ``NonZeroDigit'' etc.?  Because e.g. HexNumerals start with 0;
this allows a standalone 0 to count as a decimal numeral.)

This bit of grammar defines the form of decimal numerals, but it says
nothing about how to ``construct'' them.  By that I mean there is no
procedure associated with the grammatical rule.

By contrast, a type-theoretic definition might look something like this:

\begin{verbatim}
Declare DecimalNumeral
    := 0 | 1 | 2 | 3 | 4 | 5 | 6 | 7 | 8 | 9
    | 0 DecimalNumeral
    | 1 DecimalNumeral
    | 2 DecimalNumeral
    | 3 DecimalNumeral
    | 4 DecimalNumeral
    | 5 DecimalNumeral
    | 6 DecimalNumeral
    | 7 DecimalNumeral
    | 8 DecimalNumeral
    | 9 DecimalNumeral
    ;
\end{verbatim}

We can simplify this a bit by using a helper:

\begin{verbatim}
Declare DecimalDigit := 0 | 1 | 2 | 3 | 4 | 5 | 6 | 7 | 8 | 9 ;
Declare DecimalNumber := DecimalDigit | DecimalDigit DecimalNumber ;
\end{verbatim}

Note that these are recursive definitions.  In most grammar
meta-syntaxes, we might write something like:

\begin{verbatim}
Declare DecimalNumber := DecimalDigit+ ;
\end{verbatim}

But this relies on the grammatical meta-symbol '+', which is
unavailable.

Now something like this could be written in a grammatical meta-syntax
as part of a language definition.  But we make it part of the
language, something the user can write.  It has the effect of making
the decimal digits constructors.  An expression like ``23'' is
interpreted as applying the constructor ``2'' to the result of
applying ``3''.

Neither of these defines meaning.  For that we need additional rules,
etc.  In most programming languages, the interpretation of numerals is
implicitly defined by the compiler implementation.  In proof
assistants and some functional programming languages, this can be
specified using the language, as part of a ``prelude'' library.

In other words, type-theoretic languages can be absolutely minimal,
with no predefined semantics and hardly any syntax.  Type declarations
and definitions etc. then extend the language.

\begin{ednote}
  What makes this construction?
\end{ednote}

\begin{ednote}
  Anyway the point is the connection between type-theoretic programs
  and grammatical specification.
\end{ednote}

