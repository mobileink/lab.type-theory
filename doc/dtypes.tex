\chapter{Dependent Types}

\section{Symbols}

We're going to use numbers to explain dependent types.  We'll be
working in base 10, so we need 10 decimal digits.  Of course we can
use the standard digits 0-9, but we don't have to; we can use any
symbol at all.  For example:

{ ○ , △ , □ , ♡ , ♤ , ♢ , ♧ , ☆ , ♀ , ♂ }


Nor need we restrict our set to 10 symbols; we could stipulate that
our symbol set is infinite, but to actually use it we have to be able
to write them all down, which means we have to have a finite set, but
it can be as large as we want.  Of course the usual way to proceed is
to define a minimal set of ten symbols together with a
numeral-formation rule that allows us to construct an infinite number
of symbols from a finite set by simple concatenation of strings of the
basic ten: 1234, for example.

In fact we only need a single symbol; in what follows we'll show how
one symbol can be used to define (construct) any number system.

We'll call our basic symbol set the ``alphabet'' of our language.
We're already calling the elements of the set symbols, which means we
already have an implicit type: the concept ``symbol'' categorizes the
elements of our alphabet set.  We're going to need the term ``symbol''
later, so lets rename this category to ``symbolic'', so we can say
that our alphabet is a set of symbolics:

Symbolic :: { 0, 1, 2, 3, 4, 5, 6, 7, 8, 9 }

alternatively:

Symbolic :: { ○ , △ , □ , ♡ , ♤ , ♢ , ♧ , ☆ , ♀ , ♂ }




Example: set of Ints.  We can recognize 1 and 2 as integers and \{\}
as a set; more precisely, we can recognize the symbols because we're
familiar with the conventions governing their usage and
interpretation.  So we can recognize that the form \( \{1, 2\}\)
expresses (denotes) a set of integers.

Note that the concept ``set of integers'' is a complex concept that
depends on two more basic concepts, namely ``set'' and ``integer''.
So the more complex concept may be said to \textit{depend} on its more
basic component concepts.

The critical point here is that the combination of ``set'' and
``integer'' goes both ways.  It may seem most ``natural'' to recognize
\( \{1, 2\}\) as an example of the category (type) ``set of
integers'', but that's a matter of convention; it is equally
legitimate to recognize it as an element of the category ``integer of
sets''.  Not ``integer of sets of integers'', mind you, but just
``integer of sets''.  To see this, all that is needed is a little
reflection on the concepts involved.  First of all, the
\textit{symbols} 1 and 2 are not themselves integers; nor do the
reveal anything about the nature of integers.  The same considerations
apply to symbolic forms like \(\{1, 2\}\).  These forms are just
symbolic forms, and any form can play the role of symbol.  In
particular, sets -- or rather, set forms like \( \{1, 2\}\) -- can be
treated as symbols of anything we like.

Furthermore, integers and sets themselves -- rather than the symbolic
forms we use to refer to them -- can serve as symbols.  There is
nothing strange about calling a set an integer, of course; according
to standard set theory, that is precisely what integers are, special
kinds of sets.

The larger point: we never have direct access to things; all we have
is various descriptions.  The same thing under different descriptions
can look like different things.  That is, when we categorize things we
do not really categorize the thing itself; rather, we categorize it
\textit{under a description}.  Or, to categorize is to describe.  So a
set may ``look like'' a mere set under one description; under another
description, it comes out as an integer.

Back to the notion of ``integer of sets''.  This counts as a dependent
type.  It's a member of an \textit{indexed family of types}; other
members of this family include ``rational of sets'', ``real of sets'',
etc.

In other words, if we can for type X of Y, we can also form the type Y
of X.

What do the elements of type ``integer of sets'' look like?  Just like
sets, in fact, with one difference: the elements themselves must be
typed, which means that we have to use a \textit{constructor} for form
expressions of the type.  Let's declare that \texttt{intset} is the
constructor; then e.g. \texttt{intset \{1, 2\}} is a term of type
``integer of sets''.  (Of course we could define other syntax,
e.g. \texttt{intset 1 2}; we retain the braces to stress the concept
of integer of sets.)

So ``integer of sets'' is a type; that is, it categorizes some things.
What are those things?  Integers that depend on sets.  The type
``integer of X'' is a type family; its individuals depend on X, so
``integer of sets'' depends on ``set'', just as ``set of integers''
depends on ``integer''.

Some Int of Sets terms: intset \{1, 2\}; intset \{foo, bar\}; inset \{
\{\}, \{\{\}\} \}.  Note that these terms have something in common:
they all contain two elements.

So under the right description, these things can be treated as
integers.  For example, Int-of-Set elements of size two form an
equivalence class; the integer '2' corresponds to that class.  Note
that this makes individual integers correspond to sets of intsets.

Replace ``set'' by ``sequence'' and you get a variation on the same
theme.  An Int-of-Sequences is a kind of int that depends on the type
``sequence'' rather than set, just as ``Sequence-of-Int'' is a kind of
sequence that depends in Int (rather than, say Rational or Real).

Anyway, the point of all this is that thinking in (dependent) types is
very flexible; what counts as a type depends on how you want to
conceptualize things.

In fact, the ``base'' type of a dependent type of this kind doesn't
really matter.  A type ``X of Y'' is a new type that is distinct from
by X and Y, although it is dependent on both.  So it isn't really a
``kind'' of either X or Y.  A List-of-Int is not really a kind of
List, nor a kind of Int.  To say that one type is dependent on another
is not to say that it is a type of that kind.  A depends on B != A is
a kind of B.  E.g. List-of-Int: its ``kind'' is ``List-of-A'' where A
is a type variable (schematic var), not ``List''.  It's an instance of
the schema.

What kind a thing it is also depends on ops?  Algebra.
